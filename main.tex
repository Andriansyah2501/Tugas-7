\documentclass[conference]{IEEEtran}

\title{Analisis kekuatan Sinyal}

\author{Andrian Syah\IEEEauthorrefmark{1}, Hani Khairiyah\IEEEauthorrefmark{2}\\
\textit{Fakultas Teknologi Informasi}\\
\textit{Teknik Komputer}\\
\textit{Institut Teknologi Batam}\\
Batam, Indonesia\\
Email: \{\IEEEauthorrefmark{1}1922009, \IEEEauthorrefmark{2}1922001\}@student.iteba.ac.id}


\begin{document}
\maketitle

\begin{abstract}
    Implementasi jaringan wireless Atau yang biasa disebut dengan teknologi WLAN(Wireless Local Area Network) yang telah diatur oleh standar
    IEEE 802.11 . Jaringan WiFi digunakan untuk menghubungkan berbagai perangkat dan berbagi data.
    Analisis Wireless atau jaringan nirkabel menggunakan aplikasi InSSIDer ,
    Mengumpulkan Informasi jaringan yang ada pada sekitar lingkungan yang memiliki transmisi sinyal WiFi
\end{abstract}

\begin{IEEEkeywords}
    IEEE 802.11,InSSIDer,Wireless
\end{IEEEkeywords}

\section{Introduction}
Seiring berkembang nya zaman Teknologi semakin tidak bisa dihindarkan, 
kita sebagai manusia tidak dapat menghindar dari adanya teknologi tersebut,
banyak nya teknologi membuat banyak hal berubah sehingga menjadikan teknologi tersebut 
adalah bagian dari hidup. salah satu dari teknologi tersebut adalah teknologi jaringan nirkabel atau bisa disebut teknologi jaringan Wireless , yang dimana banyak digunakan di berbagai macam tempat  contoh nya dikampus,kafe,kedai kopi, dan lain-lain .

perkembangan Wireless ini sangat pesat sekali,karena flexible tanpa menggunakan kabel dan menghemat biaya 
, namun dari pernyataan tersebut Wireless juga banyak kelebihan dan kekurangan nya . 
Sebelumnya dianggap bahwa jaringan kabel lebih cepat dan lebih aman daripada jaringan nirkabel.
Namun peningkatan berkelanjutan pada teknologi jaringan nirkabel seperti standar jaringan Wireless
 telah membuat banyak perbedaan kecepatan dan keamanan antara jaringan kabel dan nirkabel tersebut~.
\section{Related Work}
Wireless Network atau biasa yang dikenal dengan Wi-Fi merupakan jaringan nirkabel yang digunakan
untuk mengkoneksikan dari satu perangkat ke perangkat lainnya ,
contoh nya Handphone,Laptop,Personal Computer(PC)
ke router jaringan yang sudah disambungkan internet pada port yang disediakan . biasa nya Wireless Network  
erat hubungannya dengan bidang telekomunikasi, teknologi informasi, dan teknik komputer dan jaringan 
Ada banyak jenis jaringan dan cara mengklasifikasikannya. Salah satu cara memandang jaringan adalah dari segi cakupan geografisnya yaitu:

\section{Scenario}
Pada sesi ini dijelaskan analisis jaringan yang ada di kampus Institut Teknologi Batam , ini merupakan analisis dari mahasiswapada hotspot wireless yang ada disekitar kampus . berikut ini merupakan denah lokasi yang ada pada institut teknologi Batam :
\section{Hasil dan Pembahasan}

Dari hasi analisis jaringan wireless yang dilakukan pada lingkungan kampus
institut teknologi batam menggunakan alat usb wireless penerima sinyal TP-LINK TL-WN722N yaitu pada saat kami berjalan menuju lantai 2 di area kampus iteba , jaringan akan berpindah
ke akses point terdekat dengan SSID yang sama , karena penggunaaan repeater yang ada dikampus yang menyebabkan apabila terjadi permasalahan
pada pusat jaringan atau Router Utama maka akan terjadi lost koneksi pada setiap semua repeater yang ada .
disini kami juga mencoba kecepatan internet pada SSID Iteba Student menggunakan aplikasi SpeedTest By Ookla~\cite{Ookla}.

\begin{equation}
    Rerata RSSI = \frac{Total Jumlah Nilai RSSI}{Jumlah Koordinat receiver}
    \label{rerata_rssi}
\end{equation}

\begin{table}[htbp]
    \caption{Table Analisis Pengukuran RSSI}
    \begin{center}
    \begin{tabular}{|c|c|c|c|}
        \hline
    \textbf{Nomor} & \textbf{\textit{Tinggi}}& \textbf{\textit{Receiver}}& \textbf{\textit{Rata-rata sinyal penerima}} \\
    \hline
    1 & 150cm& 25 Receiver & -53.87 dbm  \\
    \hline
    2 & 200cm& 15 Receiver & -51.98 dbm  \\
    \hline
    \multicolumn{4}{l}{$^{\mathrm{a}}$Hasil dari dana InSSIDer}
    \end{tabular}
    \label{tab1}
    \end{center}
    \end{table}

\section{Kesimpulan}
Dapat disimpulkan bahwa saat pengetesan menggunakan aplikasi InSSIDer kekuatan sinyal juga mempengaruhi transmisi data yang dimana ini menggunakan internet,
sehingga ada nya perbedaan pada saat ada nya penghalang sewaktu pengetesan dan tidak adanya penghalang saat pengetesan , beberapa keunggulan dan kekurangan dari wireless network 
sangat mungkin ada , karena pada dasarnya manusia menciptakan sesuatu hal yang belum sempurna namun dari pernyataan tersebut bahwa manusia membuat perangkat nirkabel untuk memudahkan pengaksesan 
tanpa menggunakan perantara kabel saat terkoneksi di device , perangkat hardware pada device juga mempengaruhi kekuatan sinyal , karena apabila hardware penangkap sinyal yang ada pada suatu komputer sudah usang atau sudah rusak , maka sinyal transmisi yang dihasilkan akan lebih jauh menurun.

\bibliographystyle{IEEEtran}
\bibliography{referensi.bib}

\end{document}