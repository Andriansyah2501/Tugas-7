\documentclass[conference]{IEEEtran}

\title{Analisis kekuatan}

\author{Andrian Syah\IEEEauthorrefmark{1}, Hani Khairiyah\IEEEauthorrefmark{2}\\
\textit{Fakultas Teknologi Informasi}\\
\textit{Teknik Komputer}\\
\textit{Institut Teknologi Batam}\\
Batam, Indonesia\\
Email: \{\IEEEauthorrefmark{1}1922009, \IEEEauthorrefmark{2}1922001\}@student.iteba.ac.id}


\begin{document}
\maketitle

\begin{abstract}
    Implementasi jaringan wireless (atau umum disebut sebagai jaringan WiFi) telah diatur oleh standar
    IEEE 802.11. Jaringan WiFi digunakan untuk menghubungkan berbagai perangkat dan berbagi data.
    Analisis Wireless atau jaringan nirkabel menggunakan aplikasi InSSIDer ,
    Mengumpulkan Informasi yang ada pada jaringan Wireless sekitar dan mengambil data 
    untuk dianalisis .
\end{abstract}

\begin{IEEEkeywords}
    IEEE 802.11,InSSIDer,Wireless
\end{IEEEkeywords}

\section{Introduction}
Seiring berkembang nya zaman Teknologi semakin tidak bisa dihindarkan, 
kita sebagai manusia tidak dapat menghindar dari adanya teknologi tersebut,
banyak nya teknologi membuat banyak hal berubah sehingga menjadikan teknologi tersebut 
adalah bagian dari hidup. salah satu dari teknologi tersebut adalah teknologi jaringan nirkabel
atau bisa disebut teknologi jaringan Wireless , yang dimana banyak digunakan di berbagai macam tempat 
contoh nya dikampus,kafe,kedai kopi, dan lain-lain .

perkembangan Wireless ini sangat pesat sekali,karena flexible tanpa menggunakan kabel dan menghemat biaya 
, namun dari pernyataan tersebut Wireless juga banyak kelebihan dan kekurangan nya . 
Sebelumnya dianggap bahwa jaringan kabel lebih cepat dan lebih aman daripada jaringan nirkabel.
Namun peningkatan berkelanjutan pada teknologi jaringan nirkabel seperti standar jaringan Wireless
 telah membuat banyak perbedaan kecepatan dan keamanan antara jaringan kabel dan nirkabel tersebut~.
\section{Related Work}
Wireless Network atau biasa yang dikenal dengan Wi-Fi merupakan jaringan nirkabel yang digunakan
untuk mengkoneksikan dari satu perangkat ke perangkat lainnya ,
contoh nya Handphone,Laptop,Personal Computer(PC)
ke router jaringan yang sudah disambungkan internet pada port yang disediakan . biasa nya Wireless Network  
erat hubungannya dengan bidang telekomunikasi, teknologi informasi, dan teknik komputer dan jaringan 
Ada banyak jenis jaringan dan cara mengklasifikasikannya. Salah satu cara memandang jaringan adalah dari segi cakupan geografisnya yaitu:

\section{Scenario}
*under way
\section{Hasil dan Pembahasan}
*under way
\section{Kesimpulan}
*under way
\bibliographystyle{IEEtran}
\bibliography{referensi.bib}


\end{document}